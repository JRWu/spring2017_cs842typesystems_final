\section{Related Work}
Benchmarking the performance of gradually typed languages is not a novel idea. Groups such as Takikawa et al. \cite{Takikawa:2016:SGT:2837614.2837630} have built a framework for Typed Racket which examines the performance penalties of incrementally typing modules over multiple configurations. These configurations provide a view of how developers may gradually type their programs at a macro-level. This work suggests that incrementally typing a program has significant penalties incurring slowdowns of up to 105x relative to the untyped program. Takikawa et al. discovered that by fully typing all modules within a program, the resulting program has better performance than its untyped equivalent.

TypeScript is an unsound extension of Javascript. Groups such as Rastogi et al. \cite{Rastogi:2015:SEG:2775051.2676971} have developed extensions to compile safe TypeScript code. By conducting a two-phase compilation process which introduces some runtime type information, the authors are able to enforce type safety on a subset of TypeScript code. Again, the property of type safety comes at a cost of a slowdown factor of approximately 1.15x when porting a TypeScript Compiler.